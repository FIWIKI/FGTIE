\documentclass[12pt,a4paper]{report}
\usepackage[a4paper]{geometry}
\usepackage[utf8]{inputenc}
\usepackage[spanish]{babel}
\usepackage{makeidx}
\usepackage{enumitem}
\usepackage{pgfplots}
\usetikzlibrary{positioning}
\usepackage{tikz}
\usepackage{cite}
\usepackage{textcomp}
\usepackage[T1]{fontenc}
\usepackage{graphicx}
\usepackage{grffile}
\usepackage{longtable}
\usepackage{wrapfig}
\usepackage{rotating}
\usepackage[normalem]{ulem}
\usepackage{amsmath}
\usepackage{textcomp}
\usepackage{amssymb}
\usepackage{capt-of}
\usepackage[hidelinks]{hyperref}
\usepackage{palatino}
% \pagestyle{headings}
\bibliographystyle{plain}

\addto\captionsspanish{\renewcommand{\chaptername}{Bloque}}

\title{Fundamentos de Gestión de Tecnologías de la Información}

\author{Luis Eduardo Bueso}

\date{\today}
\makeindex

\begin{document}
\begin{titlepage}
	\hfill \today

	\vspace{.2\textheight}

	\begin{center}
		{\huge\bfseries Fundamentos de Gestión de Tecnologías de la Información\par}
		\vspace{3cm}
                Luis Eduardo Bueso \par
	\end{center}

\end{titlepage}\tableofcontents

\chapter{Concepto de empresa. Elementos de una organización}
\label{chap:organizacion}
%%% Local Variables:
%%% mode: latex
%%% TeX-master: "../FGTIEapuntes"
%%% End:


\chapter{Recursos de información en la empresa}
\label{chap:recursos}
\input{sec/recursos}

\chapter{Gestión de Servicios de TI}
\label{chap:servicios}
\input{sec/servicios}

\end{document}