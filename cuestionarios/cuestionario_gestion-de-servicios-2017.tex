% \documentclass[a4paper,answers]{exam}
\documentclass[a4paper]{exam}

\usepackage[utf8]{inputenc}
\usepackage[spanish]{babel}

\usepackage{multicol}

\pagestyle{empty}

\begin{document}

\centering{\Huge Fundamentos de Gestión de Tecnologías de la
  Información en la Empresa } \vspace{0.5cm}

\centering{\Large
  Cuestionario sobre Gestión de Servicios de TI. 2017.} \vspace{1cm}

\begin{questions}

  \question La garantía del servicio está relacionada con...

  Selecciona una o más:

  \begin{checkboxes}
    \CorrectChoice Cómo de bien lo hace (lo bien que lo hace el servicio).
    \choice Lo que el servicio hace.
    \choice Las necesidades/requisitos del cliente.
    \CorrectChoice Una disponibilidad correcta.
    \CorrectChoice Una capacidad suficiente.
    \choice Un rendimiento apropiado.
  \end{checkboxes}

  \question Señale la(s) respuesta(s) respecto a los acuerdos de
  servicios de TI.

  \begin{checkboxes}
    \choice Un acuerdo entre la unidad de TI y otra parte de la misma
    organización es un \emph{UC}.
    \choice Un contrato entre un proveedor de servicios de TI y un
    tercero es un \emph{OLA}.
    \CorrectChoice Un acuerdo entre la unidad de TI y otra parte de la
    misma organización es un \emph{OLA}.
    \CorrectChoice Un contrato entre un proveedor de servicios de TI y
    un tercero es un \emph{UC}.
  \end{checkboxes}

  \question Entre las fases del \emph{Ciclo de Vida} de la
  \emph{Gestión del Servicio}, se encuentran:

  \begin{checkboxes}
    \CorrectChoice Transición
    \CorrectChoice Estrategia
    \choice Pruebas
    \choice Requisitos
  \end{checkboxes}

  \question Algunos de los parámetros/medidas habituales de las
  gestión de la capacidad son...

  \begin{checkboxes}
    \CorrectChoice Número de usuarios concurrentes.
    \choice Tiempo medio entre incidencias.
    \choice Tiempo medio de parada.
    \CorrectChoice Uso de \emph{CPU}.
  \end{checkboxes}

  \question Un usuario informa al Service Desk de un problema en su
  PC. Un representante del Service Desk determina que el equipo es
  defectuoso e indica que, según el catálogo de servicios, el equipo
  será sustituido en tres horas. ¿Qué proceso ITIL es el responsable
  de que el usuario tenga remplazado su PC en el plazo de tres horas?

  \begin{checkboxes}
    \choice Gestión de la Disponibilidad.
    \CorrectChoice Getsión del Cambio.
    \choice Gestión de la Configuración.
    \choice Gestión del Nivel de Servicio.
  \end{checkboxes}

  \question Dada las siguientes frases:

  \begin{enumerate}
  \item La gestión de problemas está mas preocupada por restablecer el
    servicio a los niveles acordados que por la calidad de la
    solución.
  \item La gestión de incidentes no persigue la búsqueda de soluciones
    definitivas, aunque puede proponerlas.
  \end{enumerate}

  ¿Cuáles son correctas?

  \begin{choices}
    \choice La 1 y la 2.
    \CorrectChoice Solo la 2.
    \choice Solo la 1.
    \choice Ninguna de las dos.
  \end{choices}

  \question Seleccione si es un \textbf{proceso} o una \textbf{función} de la fase de
  \emph{Operación del Servicio}:

  \begin{itemize}
  \item
    \begin{tabular}{p{0.4\linewidth} l}
      Gestión de Eventos &\fillin[Proceso][0.2\linewidth]
    \end{tabular}
  \item
    \begin{tabular}{p{0.4\linewidth} l}
      Gestión de Aplicaciones &\fillin[Función][0.2\linewidth]
    \end{tabular}
  \item
    \begin{tabular}{p{0.4\linewidth} l}
      Getsión de Incidencias &\fillin[Proceso][0.2\linewidth]
    \end{tabular}
  \item
    \begin{tabular}{p{0.4\linewidth} l}
      Service Desk&\fillin[Función][0.2\linewidth]
    \end{tabular}
  \item
    \begin{tabular}{p{0.4\linewidth} l}
      Gestión Técnica&\fillin[Función][0.2\linewidth]
    \end{tabular}
  \item
    \begin{tabular}{p{0.4\linewidth} l}
      Gestión del Acceso&\fillin[Proceso][0.2\linewidth]
    \end{tabular}
  \end{itemize}

  \question La Gestión de Instalaciones es responsable de gestionar...

  \begin{checkboxes}
    \CorrectChoice Hardware en un Centro de Datos o salas de
    ordenadores
    \choice Servicios de TI
    \CorrectChoice Equipo de energía y enfriamiento
    \CorrectChoice Sitios para la recuperación de servicios
  \end{checkboxes}

  \question Emparejar cada proceso con su correspondiente fase del Ciclo de Vida
  de Gestión del Servicio:

  \begin{itemize}
    \item \begin{tabular}{p{0.4\linewidth} l}
    Gestión del Conocimiento & \fillin[Transición del servicio][0.4\linewidth]
    \end{tabular}

    \item \begin{tabular}{p{0.4\linewidth} l}
    Gestión del Catálogo de Servicios & \fillin[Diseño del Servicio][0.4\linewidth]
    \end{tabular}

    \item \begin{tabular}{p{0.4\linewidth} l}
    Gestión del Cambio & \fillin[Transición del Servicio][0.4\linewidth]
    \end{tabular}

    \item \begin{tabular}{p{0.4\linewidth} l}
    Gestión de Eventos &\fillin[Operación del Servicio][0.4\linewidth]
    \end{tabular}

    \item \begin{tabular}{p{0.4\linewidth} l}
    Gestión del Portfolio de Servicios & \fillin[Estrategia del Servicio][0.4\linewidth]
    \end{tabular}

    \item \begin{tabular}{p{0.4\linewidth} l}
    Gestión de la Configuración y Activos del Servicio & \fillin[Transición del Servicio][0.4\linewidth]
    \end{tabular}

  \end{itemize}

  
\end{questions}

\end{document}