% \documentclass[a4paper,answers]{exam}
\documentclass[a4paper]{exam}

\usepackage[utf8]{inputenc}
\usepackage[spanish]{babel}

\pagestyle{empty}

\begin{document}

\centering{\Huge Fundamentos de Gestión de Tecnologías de la
  Información en la Empresa } \vspace{0.5cm}

\centering{\Large
  Cuestionario sobre la importancia de los datos en la empresa y Big Data. 2017} \vspace{1cm}

\begin{questions}

  \question El caso de estudio presentado por el representante de \emph{MS-Solutions} estuvo dedicado:

  \begin{choices}
    \choice Construcción de una base de datos de finanzas.
    \choice Aplicación de \emph{Big Data} en \emph{marketing}.
    \CorrectChoice Análisis del reto tecnológico que ha supuesto para los
    entornos financieros el adaptarse a las nuevas regulaciones para
    \emph{PYMES}.
    \choice Aplicación de técnicas de \emph{Big Data} en el ámbito médico.
  \end{choices}

  \question ¿En qué fases se emplea la mayor parte del tiempo en
  proyectos \emph{Big Data}?

  \begin{choices}
    \choice En el entendimiento del negocio y de los datos y en su
    preparación.
    \choice En el entendimiento de los datos, en su preparación y en
    la creación de los modelos.
    \choice En el entendimiento del ngocio y en la generación de los
    modelos.
    \choice En el entendimiento del negocio y en la evaluación y
    despliegue de modelos.
  \end{choices}

  \question Con qué concepto casa la siguiente definición en relación
  a la información: \emph{``es la cuantificación de los recursos asignados
      para obtenerla''}

  \begin{choices}
    \choice Oportunidad
    \choice Demanda
    \choice Valor
    \CorrectChoice Coste
  \end{choices}

  \question Los mayores problemas y retos para las empresas con
  respecto al \emph{Big Data} está en:

  \begin{choices}
    \choice La falta de presupuesto y el volumen de datos generados.
    \choice La falta de habilidades técnicas y presupuesto.
    \choice La falta de tecnología y presupuesto.
    \CorrectChoice La falta de tecnología y habilidades técnicas.
  \end{choices}

  \question En el nivel estratégico ¿cuál es el plazo en el que las
  decisiones que se toman tienen un impacto?

  \begin{choices}
    \choice Muy corto plazo
    \choice Corto plazo
    \CorrectChoice  Largo plazo
    \choice Medio plazo
  \end{choices}

  \question Hoy en día las empresas se encuentran con nuevos proyectos
  que requieren análisis de grandes volúmenes de datos para extracción
  de conocimiento que les permita moverse en entornos muy
  competitivos. El gran reto al que se tienen que enfrentar es:

  \begin{choices}
    \choice Transformación de ficheros de \emph{Excel} a
    \emph{SQLServer}.
    \choice Eficiencia en las respuestas a los clientes.
    \choice Captura de fuentes adecuadas, transformación en una base
    de datos \emph{SQL} y consulta.
    \CorrectChoice Captura de fuentes adecuadas, eficiencia y eficacia.
  \end{choices}

  \question ¿Cuál de los siguientes niveles hace referencia a las
  decisiones que se toman en el día a día?

  \begin{choices}
    \choice Estratégico
    \CorrectChoice Operativo
    \choice Discrecional
    \choice Táctico
  \end{choices}

  \question ¿Cuántos niveles de decisión tenemos en la empresa cuando
  hablamos de la toma de decisiones en estructura?

  \begin{choices}
    \choice 5
    \CorrectChoice 3
    \choice 4
    \choice Ninguna de las anteriores.
  \end{choices}

  \question La gran diferencia con las que nos encontramos al realizar
  un proyecto real de empresa con respecto a los trabajos que hemos
  desarrollado en la universidad es:

  \begin{choices}
    \choice Quien lo hace no recibe dinero.
    \choice Responsabilidad de quien lo hace.
    \CorrectChoice La dimensión.
    \choice No hay diferencias.
  \end{choices}

  \question ¿Cuáles son las 5 \emph{V} que definen el \emph{Big Data}?

  \begin{choices}
    \CorrectChoice Volumen, Veradidad, Velocidad, Valor y Variedad.
    \choice Volumen, Velocidad, Valor, Visión y Veracidad.
    \choice Volumen, Veracidad, Velocidad, Valor y Vehicular.
    \choice Volumen, Valor, Variedad, Viscosidad y Volatilidad.
  \end{choices}
  
\end{questions}

\end{document}