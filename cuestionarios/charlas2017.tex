\documentclass[a4paper,answers]{exam}
% \documentclass[a4paper]{exam}

\usepackage[utf8]{inputenc}
\usepackage[spanish]{babel}

\usepackage{multicol}

\pagestyle{empty}

\begin{document}

\centering{\Huge Fundamentos de Gestión de Tecnologías de la
  Información en la Empresa } \vspace{0.5cm}

\centering{\Large
  Cuestionario de sobre los tres seminarios. 2017} \vspace{1cm}
\begin{questions}

  \question Empareja las siguientes fases o etapas de la planificación
  estratégica con la explicación de cada una de ellas:

  \begin{multicols}{3}
    Análisis externo \\
    Estrategias  \\
    Visión compartida\\
    Análisis interno\\
    Objetivo \\
    Misión/Valores
  \end{multicols}

  \begin{itemize}
  \item
    \begin{tabular}{p{0.6\linewidth} r}
      ¿Qué podemos hacer? & \fillin[Análisis interno][0.25\linewidth]
    \end{tabular}
  \item
    \begin{tabular}{p{0.6\linewidth} r}
      ¿Cómo lo vamos a hacer? & \fillin[Estrategias][0.25\linewidth]
    \end{tabular}
  \item
    \begin{tabular}{p{0.6\linewidth} r}
      ¿Qué vamos a hacer para conseguirlo? & \fillin[Objetivo][0.25\linewidth]
    \end{tabular}
  \item
    \begin{tabular}{p{0.6\linewidth} r}
      ¿Qué somos? ¿Qué hacemos? & \fillin[Misión/Valores][0.25\linewidth]
    \end{tabular}
  \item
    \begin{tabular}{p{0.6\linewidth} r}
      ¿Qué nos piden que hagamos?¿Qué encargo tenemos?  &
                                                          \fillin[Análisis externo][0.25\linewidth]
    \end{tabular}
  \item
    \begin{tabular}{p{0.6\linewidth} r}
      ¿Qué queremos ser en el futuro? & \fillin[Visión
                                        compartida][0.25\linewidth]
    \end{tabular}
  \end{itemize}
  


  
\question La función directiva de planificar, que consiste en hacer
  el plan o proyecto de actuación, o proyectar una unidad o una
  institución hacia el futuro deseado, incluye entre su actuaciones:

  \begin{enumerate}
  \item garantizar el uso eficaz de los recursos.
  \item concretar las partes de la organización y definir su
    estructuración.
  \item comunicación entre las personas que ocupan puestos de base y
    toda la cadena de mando y entre los distintos niveles de
    dirección.
  \end{enumerate}

  
  Seleccione una opción:
  
  \begin{choices}
    \choice 1 y 3 son correctas
    \choice 1 y 2 son correctas
    \CorrectChoice Sólo 1 es correcta
    \choice Sólo 2 es correcta
  \end{choices}

  
\question Un ingeniero informático puede desarrollar su actividad en
  cualquier proceso de negocio de TI en una organización:

  \begin{enumerate}
  \item Tanto a nivel de gestión de negocio  como a nivel técnico.
  \item Tanto a nivel de diseño , como en desarrollo, o en servicios y operaciones.
  \item Pero no como consultor de TI.
  \end{enumerate}

  
  Seleccione una opción:
  
  \begin{choices}
    \choice Las tres son ciertas.
    \CorrectChoice Verdad la 1 y 2 pero no la 3.
    \choice Verdad  la 1 pero no  2 y 3.
    \choice Verdad  la  2 pero no 1 y 3.
    \choice Verdad la  3 pero no  1 y 2.
  \end{choices}

\question La ingeniería informática es la rama de la ingeniería que
  aplica los fundamentos de la ciencia de la computación, la
  ingeniería electrónica y la ingeniería de software, para el
  desarrollo de soluciones integrales de cómputo y comunicaciones, que
  utilizando la tecnología de la información permite la captura de
  datos, su registro, su proceso para obtener información, su
  almacenamiento y su transporte y distribución para su uso.

  
  Selecciones una opción:

  \begin{choices}
    \CorrectChoice Verdadero
    \choice Falso
  \end{choices}

\question ¿Es fácil para una empresa contratar a un ingeniero
  informático con el perfil adecuado para un puesto de trabajo y no
  equivocarse?

  \begin{enumerate}
  \item SI, le es suficiente con contratar a un  ingeniero con su título correspondiente.
  \item SI, pero   tendrá que mantener una entrevista muy profunda sobre lo que se espera de él  y fiarse de lo que diga.
  \item SI, pero le será más fácil si alguna organización certifica que tiene las competencias necesarias para  el perfil pedido ya que dicha organización lo ha verificado previamente.
  \end{enumerate}

  
  Selecciones una:
  
  \begin{choices}
    \choice Verdad la 1 y 2 pero no la 3.
    \CorrectChoice Verdad la 3 pero no 1 y 2.
    \choice Las tres son ciertas.
    \choice Verdad  la  2 pero no 1 y 3.
    \choice Verdad  la 1 pero no  2 y 3 .
  \end{choices}

\question Las cinco áreas que determina e-CF son:
  \textbf{Planificar} (\emph{plan}), \textbf{Construir}
  (\emph{build}), \textbf{Ejecutar} (\emph{run}), \textbf{Facilitar}
  (\emph{enable}) y \textbf{Gestionar} (\emph{manage}).

  En relación con las áreas del e-CF, cual afirmación es correcta:

  \begin{enumerate}
  \item \emph{Planificar}, \emph{construir} y \emph{ejecutar}
    son las áreas centrales, mientras que \emph{facilitar} y
    \emph{gestionar} son áreas transversales relacionadas con las
    anteriores.
  \item \emph{Planificar} y \emph{facilitar} representan áreas
    estratégicas en las compañías, y es donde se conciben, diseñan y
    crean productos, servicios, acciones y políticas.  
  \item \emph{Construir} está relacionado con el desarrollo e
    implementación de productos, servicios y/o soluciones, y
    \emph{ejecutar} se focaliza en la provisión, soporte y
    mantenimiento de los productos, servicios y/o soluciones
    desarrollados y desplegados.
  \item \emph{Gestionar} representa la administración diaria y la mejora que se lleva a cabo en las empresas.  
  \end{enumerate}

  Seleccione una opción:
  
  \begin{choices}
    \CorrectChoice 1, 2, 3 y 4 son correctas.
    \choice 2 y 3 son correctas.
    \choice 1 y 2 son correctas.
  \end{choices}
  
  \question ¿Qué es una competencia?

  \begin{enumerate}
  \item Las competencias pueden ser vistas como una combinación de
    conocimientos, habilidades y actitudes. Como tal, las competencias
    van mucho más allá de los conocimientos teóricos.
  \item También se refieren a los tipos de comportamiento que muestran
    las personas, qué experiencia tienen y cómo pueden aplicar sus
    conocimientos.
  \end{enumerate}

  Seleccione una opción:

  \begin{choices}
    \choice Solo es correcta la 2.
    \choice Solo es correcta la 1
    \CorrectChoice Son correctas la 1 y la 2.
  \end{choices}

  \question El marco Europeo de Competencias es una herramienta que
  puede usarse a nivel interncional y que está dirigida a todos los
  que de una u otra forma utilizan las TIC.

  Seleccione una opción:

  \begin{choices}
    \CorrectChoice Verdadero
    \choice Falso
  \end{choices}

  \question A la hora de establecer el \emph{e-Competence Framework}
  se instituyeron una serie de principios que concretan la filosofía
  adoptada en la construcción y posteriores versiones de \emph{e-CF} y
  que guían su desarrollo y aplicación. Estos principios son los
  siguientes:

  \begin{enumerate}
  \item \emph{e-CF} \textbf{debe ser un facilitador}, ya que está
    pensado como una herramienta para potenciar a los usuarios, no
    para limitarlos.
  \item \textbf{Una competencia no es una definición de un puesto de
      trabajo}, aunque puede ser parte de la misma. Por poner un
    ejemplo, \emph{``Gestión de ventas''} es una competencia de un
    \emph{``Director de ventas''} pero no es lo único que define el
    perfil de dicho director.
  \item \emph{e-CF} \textbf{es neutral y de libre utilización}, siendo
    desarrollado y mantenido mediante un proceso de acuerdo de
    múltiples partes interesadas en toda la UE, bajo la supervisión
    del Comité Europeo de Normalización (CEN).
  \end{enumerate}

  Seleccione una opción:

  \begin{choices}
    \choice 3 es correcta.
    \choice 1 es correcta.
    \CorrectChoice 1, 2 y 3 son correctas.
  \end{choices}

  \question Las competencias profesionales solo pueden ser evaluadas y
  certificadas por la universidades.

  Seleccione una opción:

  \begin{choices}
    \choice Verdadero
    \CorrectChoice Falso
  \end{choices}
  
\end{questions}

\end{document}